% Compiled and edited by Brian Bi
\documentclass[10pt]{extarticle}
\setlength{\parindent}{0.0in}
\usepackage{xeCJK}
\usepackage{fontspec, xunicode, xltxtra}
\usepackage{amsmath}
\usepackage{amssymb}
\usepackage{amsfonts}
\usepackage{pifont}
\usepackage{hyperref}
\usepackage[a5paper,twoside=true,top=15mm,bottom=10mm,left=10mm,right=10mm]{geometry}
\usepackage{fancyhdr}
\pagestyle{fancy}
\fancyhead{}
\fancyfoot{}
\fancyhead[LE,RO]{\thepage}
\fancyhead[LO]{\leftmark}
\fancyhead[RE]{\rightmark}

\usepackage{listings}
\makeatletter
\lst@CCPutMacro\lst@ProcessOther {"2D}{\lst@ttfamily{-{}}{-{}}}
\@empty\z@\@empty
\makeatother
\usepackage{color}
\newfontfamily\mono{Consolas}
\newfontfamily\timesnew{Times New Roman}
\lstset {
  basicstyle = \small\mono,
  columns = fixed,
  extendedchars=false,
  language = C++,
  tabsize = 4,
  frame = single,
  breaklines = true,
  breakindent = 1.1em,
  numbers=left,
  stringstyle=\mono,
  numberstyle=\footnotesize\mono,
  basewidth={0.5em, 0.4em},
}

\newcommand\Cross{%
 \makebox[1.0em][c]{%
 \makebox[0pt][c]{\raisebox{0em}{\ding{56}}}}~~}%

\newcommand\Warning{%
 \makebox[1.0em][c]{%
 \makebox[0pt][c]{\raisebox{0.1em}{\tiny!}}%
 \makebox[0pt][c]{$\bigtriangleup$}}~~}%

\newcommand\Tick{%
 \makebox[1.0em][c]{%
 \makebox[0pt][c]{\raisebox{0em}{$\checkmark$}}}~~}%

\newcommand\Require[1]{\ref{#1} \nameref{#1} }

\newcommand*\newevenpage{%
  \clearpage
  \ifodd\value{page}\hbox{}\newpage\fi
}

\newcommand{\BookNo}{2}
\newcommand{\BookTitle}{Number Theory \\[0.2cm] Linear Algebra \\[0.4cm] Combinatorics}
\begin{document}
\begin{titlepage}
\begin{center}
% Upper part of the page
\includegraphics[width=0.15\textwidth]{../shared/njulogo.pdf}\\[1cm]
\textsc{\huge Nanjing University}\\[3cm]
\timesnew{\huge ACM-ICPC Codebook \BookNo}\\[0.5cm]
% Title
%\HRule \\[0.4cm]
{ \Huge \bfseries \BookTitle}\\[0.4cm]
%\HRule \\[1.5cm]


\vfill

% Bottom of the page
{\large \today}

\end{center}

\end{titlepage}

\setmainfont{Times New Roman}
\setlength{\parskip}{0.0in}
\tableofcontents
\setlength{\parskip}{0.1in}
\newevenpage
% New codebook
\section{Number Theory}
\subsection{Modulo operations}
\subsubsection{Modular exponentiation (fast power-mod)} \label{powmod}
Calculate $b^e \bmod m$. \par
\Warning Cannot be performed on \lstinline|long long|, unless use \Require{mulmod}. \par
\textbf{Time complexity:} $O(\log e)$ \par
\lstinputlisting[language=c++]{number-theory/powmod.cpp}

\subsubsection{Mathematical modulo operation}
The result has the same sign as divisor. \par
\lstinputlisting[language=c++]{number-theory/mathmod.cpp}

\subsubsection{Modular multiplication on long long} \label{mulmod}
Calculate $ab \bmod m$, where $a, b, m$ are \lstinline|long long| integers. \par
\Warning $a, b, m$ must be non-negative. \par
\textbf{Time complexity:} $O(\log b)$ \par
\lstinputlisting[language=c++]{number-theory/mulmod.cpp}

\subsection{Extended Euclidian algorithm} % Stanford'
Solve $ax + by = g = \gcd(a,b)$ w.r.t. $x, y$. \par
If $(x_0, y_0)$ is an integer solution of $ax + by = g = \gcd(x,y)$, then every integer solution of it can be written as $(x_0 + kb', y_0 - ka')$, where $a' = a/g$, $b' = b/g$, and $k$ is arbitrary integer. \par
\Warning $x$ and $y$ must be positive. \par
\textbf{Usage:} \\[0.1cm]
\begin{tabular}{p{4cm} p{7cm}}
  \lstinline|exgcd(a, b, g, x, y)| & Find a special solution to $ax + by = g = \gcd(a, b)$. \\
\end{tabular} \par
\textbf{Time complexity:} $O(\log \min \{a,b\})$ \par
\lstinputlisting[language=c++]{number-theory/exgcd.cpp}

\subsubsection{Modular multiplicative inverse}
An integer $a$ has modular multiplicative inverse w.r.t. the modulus $m$, iff $\gcd(a, m) = 1$. Assume the inverse is $x$, then
$$ ax \equiv 1 \mod m \text{.}$$
Call \lstinline|exgcd(a, m, g, x, y)|, if $g = 1$, $x + km$ is the modular multiplicative inverse of $a$ w.r.t. the modulus $m$. \par
\lstinputlisting[language=c++]{number-theory/minv.cpp} \par
Or, by Fermat's little theorem ($a^p \equiv a \mod p$), when $m$ is a prime, the multiplicative can also be written as 
$a^{-1} = \left(a^{p-2} \bmod p\right) \text{.}$

\subsection{Primality test (Miller-Rabin)}
Test whether $n$ is a prime. \par
The array \lstinline|a[]| (excluding sentinel, e.g. \lstinline|LLONG_MAX|) should be \\
\begin{tabular}{p{6cm} p{5cm}}
  \lstinline|{2}| & when $n < 2,047$. \\
  \lstinline|{2, 7, 61}| & when $n < 4,759,123,141$ ($2^{32}$) . \\
  \lstinline|{2, 3, 5, 7, 11}| & when $n < 2.1 \times 10^{12}$. \\
  \lstinline|{2, 325, 9375, 28178, 450775, 9780504, 1795265022}| & when $n < 2^{64}$.
\end{tabular} \par
\Warning When $n$ exceeds the range of \lstinline|int|, the mul-mod and pow-mod operations should be rewritten. \par
\textbf{Requirement:} \\
\Require{powmod} \par
\textbf{Time complexity:} $O(\log n)$ \par
\lstinputlisting[language=c++]{number-theory/miller-rabin.cpp}

\subsection{Sieve of Eratosthenes}

\subsection{Chinese remainder theorem}


\subsection{Quadratic residue}
\subsubsection{Legendre symbol}
For non-negative integer $a$ and \textbf{odd} prime $p$, the Legendre symbol is defined as
$$ \left( \frac{a}{p} \right) = \begin{cases} 0 & \text{if } a \mid p  \\ 1 & \text{if } a \nmid p \text{ and } a \text{ is a quadratic residue modulo } p \\ p-1 & \text{if } a \nmid p \text{ and } a \text{ is a quadratic non-residue modulo } p \end{cases} $$
Call \lstinline|powmod(a, (p-1)/2, p)| to calculate Legendre symbol.
\section{Linear Algebra}
\subsection{Modular exponentiation of matrices}
Calculate $b^e \bmod modular$, where $b$ is a matrix. The modulus is element-wise.\par
\textbf{Usage:} \\[0.1cm]
\begin{tabular}{p{3cm} p{8.5cm}}
  \lstinline|n| & Order of matrices. \\
  \lstinline|modular| & The divisor in modulo operations. \\
  \lstinline|m_powmod(b, e)| & Calculate $b^e \bmod modular$. The result is stored in \lstinline|r|. \\
\end{tabular} \par
\textbf{Time complexity:} $O(n^3 \log e)$ \par
\lstinputlisting[language=c++]{linear-algebra/matrix_powmod.cpp}

\section{Combinatorics}
\subsection{P\'olya enumeration theorem}
\end{document}
